\chapter{Implementation}

Topics: disk blocks, hashing, and B+ trees.

Databases store data persistently on a hard disk, and since most hard
disks in use today are revolving magnetic disks, databases are highly
optimized for this.

Revolving magnetic disks, as opposed to solid state disks, have a
physical disk that spins under a moving read/write head.  Such disks
are organized into sectors (a slice of the disk), and each sector is
further divided into blocks of a fixed number of bytes.  In order to
read to or write a block in a sector of the spinning disk, both the
disk and head must move, which is the bottleneck in the speed of a
drive.

Databases are optimized for this scenario, in that they minimize the
number of blocks to be read from the drive, thus minimizing the amount
of disk and head movement.

\section{B+ Trees}

The biggest optimization to minimize block retrievals is to store as
much information within a block as possible.  A B+ tree is a tree with
a large branching factor wherein all data nodes are leafs.

In a B+ tree, a branching factor $b$ is specified and largely
determines the structure of the tree.  The branching factor $b$ is an
upper bound on the number of children each internal node can have.

The smallest possible B+ tree contains 2 nodes: the root node with one
child data node.  

If at any point the number of nodes pointed to by the root surpasses
the branching factor $b$, the nodes are split between two new internal
nodes to which the root now points.  Note that these nodes may be data
nodes as in the base case, or they may be internal nodes which may
themselves point to internal nodes or data nodes.



