\chapter{Relational Algebra}

Given a relation $(S,R)$, we define a formalism for retrieving
information.

\section{Select}

The \emph{selection} operator denoted \Select{} takes two paramters: a
set of attributes in $S$ and a relation $(S,R)$, and returns a new
relation $(S',R')$ in which $S'$ contains only the specified
attributes of $S$ and $R'$ contains modified tuples with only values
corresponding to the attributes of $S'$.  We denote this as
\Select{X}Y where X is the set of attributes to keep and Y is the
relation.

\section{Project}

The \emph{projection} operator denoted \Project{} takes two
parameters: a set of conditions on the attributes of $S$ and a
relation $(S,R)$, and returns a new relation $(S,R')$ where $R'
\subseteq R$ is the set of all tuples that meet the specified
conditions.  We donte this as \Project{X}Y where X is the set of
conditions and Y is the relation.

\section{Renaming Attributes}

It is often useful to rename attributes and save intermediate
relations.  We do this with the \GoesTo operator as in the following
example.

Stuff(Noms,NotNoms) \GoesTo \Select{Edibles,Inedibles} Items

This example selects the Edibles and Inedibles attributes of the Items
relation, and saves them to a new relation Stuff and rename the
attributes to Noms and NotNoms.

\section{Cartesian Product}

The \emph{Cartesian Product} binary operator denoted \CartesianProduct
takes relations $(S,R)$ and $(S',R')$ and returns the relation $(S
\cup S',R'')$ where $R''$ is a set of tuples $\{r r' | r \in R, r' \in
R' \}$.

\section{Natural Join}

The \emph{natural join} binary operator denoted \NaturalJoin takes
relations $(S,R)$ and $(S',R')$ where $J = S \cap S' \neq \emptyset$,
and returns $(S \cup S',R'')$ where $R''$ is the set of tuples $\{r r'
| r \in R, r' \in R', r(J) = r'(J) \}$.

\section{Join}

The \emph{join} operator denoted \Join{} takes relations $(S,R)$,
$(S',R')$, and a function $f$, and returns the relation $(S \cup
S,R'')$ where $R''$ is the set of tuples $\{ r r' | r \in R, r' \in
R', f(r,r') = true \}$.

For example:

CALL
\Join{CALL.portid = CALL\_FORWARD\_NUMBERS.portid}
CALL\_FORWARD\_NUMBERS

Joins CALL and CALL\_FORWARD\_NUMBERS when CALL.portid =
CALL\_FORWARD\_NUMBERS.portid.

\section{Aggregate Function}

The \emph{Aggregate Function} takes a set of attributes from the
relation, a function list, and a relation and returns the result of
applying the function to the tuples of the relation and grouping by
the given attributes.  Available functions are SUM, AVERAGE, MINIMUM,
MAXIMUM, COUNT.

For example:

lineid \AggregateFunction{count scode} SERVICE\_SUBSCRIBERS

Returns the count of unique values for attribute scode, grouped by
lineid on relation SERVICE\_SUBSCRIBERS.
