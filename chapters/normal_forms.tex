\chapter{Normal Forms}

\section{First Normal Form (1NF)}

\begin{definition}

A \emph{superkey} of a relation $S,R$ is a set of attributes $X$ such
that $t_1(X) = t_2(X)$ if and only if $t_1 = t_2$.  Such attributes
are said to be \emph{prime}.  

A superkey is said to be \emph{minimal} if it has the least number of
attributes required to meet this condition.  Such a minimal superkey
is called a candidate key.

\end{definition}

\begin{definition}

A set of relations is in \emph{First Normal Form} if every relation
has a minimal superkey.

\end{definition}

\section{Second Normal Form (2NF)}

\begin{definition}

A \emph{partial dependency} is a dependency of a non-prime attribute
on a proper subset of a candidate key.

\end{definition}

\begin{definition}

A set of relations is in \emph{Second Normal Form} if it is in 1NF and
it contains no partial dependencies.

\end{definition}

\section{Third Normal Form (3NF)}

\begin{definition}

A \emph{trivial dependency} is a dependency $X \rightarrow Y$ where $Y
\subseteq X$.

\end{definition}

\begin{definition}

A \emph{transitive dependency} is a dependency inferred from the
transitive axiom.  If $X \rightarrow Y$ is a transitive dependency, we
say $Y$ is \emph{transitively dependendent} on $X$, otherwise $Y$ is
\emph{directly dependent} on $X$.

\end{definition}

\begin{definition}

A set of relations is in \emph{Third Normal Form} if it is in 2NF and
all functional dependencies $X \rightarrow Y$ are trivial, or X is a
superkey, or all attributes $a \in (X - Y)$ are prime.

\end{definition}

\section{Boyce-Codd Normal Form (BCNF)}

\begin{definition}

A set of relations is in \emph{Boyce-Codd Normal Form} if it is in 2NF
and all functional dependencies $X \rightarrow Y$ are trivial or X is
a superkey.

\end{definition}
