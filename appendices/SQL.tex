\chapter{SQL}

SQL is a Data Definition Language (DDL) and a Data Manipulation
Language (DML), which means that SQL is suitable for both defining and
manipulating large amounts of data.

\section{SELECT}

The \texttt{SELECT} operation in SQL is a combination of the
\emph{select} and \emph{project} operations in relational algebra.
The basic syntax is:

\texttt{SELECT columns FROM table WHERE condition;}

Where \texttt{columns} are columns in table \texttt{table}, and
\texttt{condition} is a boolean condition like \texttt{col1 > 5}.

For example, given the following \texttt{coders} table schema:

\begin{tabular}{|c|}
  
  \hline
  {\bf name} \\
  \hline
  codingAbility \\
  \hline
  birthDate \\
  \hline
  
\end{tabular}

We could run the following query:

\texttt{SELECT name, birthDate FROM coders \newline
WHERE codingAbility > 9000;}

Which would return all coders whose codingAbility is over 9000.

There are many extra options that can be specified on a
\texttt{SELECT} query, the most useful of which is easily the
\texttt{ORDER BY} option:

\texttt{SELECT name, birthDate FROM coders \newline
ORDER BY codingAbility DESC;}

Which would return the names and birthDates of the coders in
descending order of their codingAbility.

\section{JOIN}

A \texttt{JOIN} in SQL is just like a join in relational algebra.
There are many joins in SQL, but by far the most common is the
\texttt{INNER JOIN}, which only returns rows on which the join
condition can be met.

Given the \texttt{coders} table described in the previous section, and
the \texttt{coderAlias} table whose schema is as follows:

\begin{tabular}{|c|}
  
  \hline
  {\bf name} \\
  \hline
  alias \\
  \hline
  
\end{tabular}

We can make the following query:

\texttt{SELECT alias, birthDate \newline
FROM coders INNER JOIN coderAlias \newline
ON coders.name = coderAlias.name;}

Which would return the alias and birthDate of all coders who have a
row in both the \texttt{coders} and \texttt{coderAlias} tables.

\section{Aggregate Functions}

Just like in relational algebra, SQL has the concept of aggregate
functions such as: \texttt{MIN}, \texttt{MAX}, and \texttt{COUNT}.

A query to count the number of coders with each birthDate is as
follows:

\texttt{SELECT birthDate, COUNT(*) AS count FROM coders \newline
GROUP BY birthDate;}

The \texttt{GROUP BY} statement is crucial in this case, to aggregate
the data by coder.  Without the \texttt{GROUP BY} statement, the query
would simply count how many birthDates were associated to each coder,
which would be 1 in all cases because each coder has a unique row in
the table.

Note that this query also renames the \texttt{COUNT(*)} column of the
results as \texttt{count}, for convenience.
